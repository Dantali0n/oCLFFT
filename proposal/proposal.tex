\documentclass[conference]{IEEEtran}
\IEEEoverridecommandlockouts
% The preceding line is only needed to identify funding in the first footnote.
% If that is unneeded, please comment it out.
%\usepackage[backend=bibtex,style=verbose-trad2]{biblatex}

\usepackage{adjustbox}
\usepackage{algorithmic}
\usepackage{amsmath,amssymb,amsfonts}
%\usepackage[backend=bibtex,style=ieee]{biblatex}
%\usepackage{bookmark}
\usepackage{booktabs}
\usepackage{caption}
\usepackage{cite}
\usepackage{color}
\usepackage[inline]{enumitem}
\usepackage{float}
\usepackage[T1]{fontenc}
%\usepackage{fontspec}
\usepackage{footnote}
\usepackage{graphicx}
\usepackage[colorlinks=true,citecolor=blue]{hyperref}
\usepackage{inputenc}[utf8]
\usepackage{listings}
\usepackage{textcomp}
\usepackage[flushleft]{threeparttable}
\usepackage{subcaption}
\usepackage{xcolor}
\usepackage{siunitx}

% \usepackage{newunicodechar}
% \newunicodechar{°}{\degree}

\title{Fast Fourier Transform%\\
%{\footnotesize \textsuperscript{*}Note: Sub-titles are not captured in Xplore 
% and should not be used}
%\thanks{Identify applicable funding agency here. If none, delete this.}
}

\pagenumbering{arabic}
\pagestyle{plain}

% Ensure decimal numbering in table  of contents
\renewcommand{\thesection}{\arabic{section}}
\renewcommand{\thesubsection}{\arabic{section}.\arabic{subsection}}
\renewcommand{\thesubsubsection}{\arabic{section}.\arabic{subsection}.\arabic{subsubsection}}

% ensure decimal numbering for sub sections
\makeatletter
\renewcommand{\@seccntformat}[1]{\csname the#1\endcsname.\quad}
\makeatother

% ------------------------------------------------------------------------%
% Proper Python Syntax Highlighting                                       %
% Author: redmode
% https://tex.stackexchange.com/questions/83882/how-to-highlight-python   %
% -syntax-in-latex-listings-lstinputlistings-command#83883                %
% ----------------------------------------------------------------------- %

% Default fixed font does not support bold face
\DeclareFixedFont{\ttb}{T1}{txtt}{bx}{n}{8} % for bold
\DeclareFixedFont{\ttm}{T1}{txtt}{m}{n}{8}  % for normal

% Custom colors
\definecolor{deepblue}{rgb}{0,0,0.5}
\definecolor{deepred}{rgb}{0.6,0,0}
\definecolor{deepgreen}{rgb}{0,0.5,0}

% Python style for highlighting
\newcommand\pythonstyle{
	\lstset{
		language=Python,
		basicstyle=\ttm,
		showstringspaces=false,
		tabsize=4,
		aboveskip=0.2cm,
		belowskip=0.2cm,
		otherkeywords={self},             % Add keywords here
		keywordstyle=\ttb\color{deepblue},
		emph={MyClass,__init__},          % Custom highlighting
		emphstyle=\ttb\color{deepred},    % Custom highlighting style
		stringstyle=\color{deepgreen},
		frame=tb,                          % Any extra options here
		prebreak=\textbackslash,
		linewidth=8.85cm,
		breaklines=true,
	}
}

% Python environment
\lstnewenvironment{python}[1][] {
	\pythonstyle\lstset{#1}
}{}

% Python for inline
\newcommand\pythoninline[1]{{\pythonstyle\lstinline!#1!}}

% Python for external file
\newcommand\pythonexternal[2][]{{\pythonstyle\lstinputlisting[#1]{#2}}}

% ----------------------------------------------------------------------- %

% Bash style for highlighting
\newcommand\bashstyle{
	\lstset{
		language=Bash,
		basicstyle=\ttm,
		showstringspaces=false,
		tabsize=2,
		aboveskip=0.2cm,
		belowskip=0.2cm,
		prebreak=\textbackslash,
		extendedchars=true,
		mathescape=false,
		linewidth=8.85cm,
		breaklines=true
	}
}

% Bash environment
\lstnewenvironment{bash}[1][] {
	\bashstyle\lstset{#1}
}{}

% Bash for inline
\newcommand\bashinline[1]{{\bashstyle\lstinline!#1!}}

% Bash for external file
\newcommand\bashexternal[2][]{{\bashstyle\lstinputlisting[#1]{#2}}}

\onecolumn

\begin{document}

\section{Fast Fourier Transform (FFT)}
One of my favorite general purpose algorithms is Discrete Fourier Transform
(DFT), however, this algorithm has a high computational complexity of
$\mathcal{O}(N^2)$. Combined with often accumulating round-off-errors this algorithm
is rarely used in practice. Luckily, its twin brother Fast Fourier Transform
(FFT) is the basis for many modern scientific applications and typically has a
lower complexity of $\mathcal{O}(N\log(N))$. This algorithm manages to transform
data from its original domain into a frequency domain often providing complex
numbers with both a real and imaginary component\footnotemark[1].

\footnotetext[1]{Sometimes called IQ in sampling systems although that implies
the Q is always at a $90\si{\degree}$ phase difference from I while the
imaginary component could be at any angle. Together the real and imaginary
component are still used to create a 2D vectors just like in IQ sampling
systems.}

Many different derivatives of the FFT algorithm exist all with different
properties and restrictions. For some it might seem less interesting to
implement some of these derivatives and model their results, however, it is
important to know ones limits. Knowing I have substantial difficulty with most
mathematical subjects, I think this will be an excellent balance between an
interesting subject and mathematical complexity. Perhaps two or three projects
into the future will I be able to theorizes potential optimizations for FFT
algorithms as its lower-bound computational complexity is still an ongoing
area of research.

\subsection{Envisioned approach}
Having no previous experience implementing FFT algorithms, a first step is to
implement a straight-forward relatively simple sequential FFT algorithm. Perhaps
even start with a simple Discrete Cosine Transform (DCT)\footnotemark[2] or DFT
as these building blocks can be expanded upon to create FFT algorithms.

\footnotetext[2]{DCT and Discrete Sine Transform (DST) are not considered valid
FFT algorithms as they only compute real or imaginary values respectively.}

Afterwards an FFT algorithm with specific properties will be chosen to
be implemented such as the Cooley-Tukey algorithm\cite{Cooley1965}. This
algorithm will be optimized for the specific hardware platform taking properties
such as cache size, vector size and locality into account. Furthermore, this
implementation is later adapted to use multi-threading and adapted to be
computed on a GPU. Inspiration as to how to realize these transformations might
be sought from libraries such as FFTW\cite{FFTW} and cuFFT\cite{cufft}. In
addition to using these for inspiration they can be used for output validation
and performance evaluation as well.

\bibliographystyle{IEEEtran}
\bibliography{bibliography}

\end{document}